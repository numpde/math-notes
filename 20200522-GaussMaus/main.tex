\documentclass[12pt]{article}

\usepackage{amsmath, amsfonts}

%
\usepackage{fullpage}
\setlength{\parindent}{0cm}

% For editing purposes:
%\usepackage[margin=10pt]{geometry}

% Macros
\newcommand{\IP}{\mathbb{P}}
\newcommand{\IE}{\mathbb{E}}
\newcommand{\IN}{\mathbb{N}}
\newcommand{\norm}[2]{\|#1\|_{#2}}
\newcommand{\scalar}[2]{\langle#1\rangle_{#2}}
\newcommand{\from}{\colon}


\begin{document}
	
	\section*{A note on [KB/2020]}
	
	%
	
	\subsection*{Introduction}
	
	%
	
	
	Let $V \hookrightarrow H \cong H' \hookrightarrow V'$
	be a Gelfand triple of separable Hilbert spaces with dense and continuous embeddings.
	%
	We write $\norm{\cdot}{X}$ for the norm on $X$,
	and
	$\scalar{\cdot, \cdot}{}$ for the duality pairing on $V' \times V$.
	%
	Let $V_n \subset V$ be nontrivial finite-dimensional subspaces, 
	parameterized by $n \geq 1$.
	%
	Let $Q_n \from V' \to V_n$ denote the $H$-orthogonal projection.
	Let $P_n \from V' \to V_n$ denote the $V$-orthogonal projection.
	%
	We assume that
	\begin{align}
		\label{e:P}
		%
		\forall v \in V:
		\quad
		\norm{v - P_n v}{V} = o(\norm{v}{V})
		\quad\text{as}\quad
		n \to \infty
		.
	\end{align}
	
	We abbreviate $\norm{\cdot}{V} := \norm{\cdot}{\mathcal{L}(V)}$ 
	for the operator norm,
	for example
	$\norm{Q_n}{V} := \norm{Q_n|_V}{\mathcal{L}(V)}$.
	%
	%
	Recall [S/2006]
	that for any nontrivial idempotent operator $Q \from V \to V$ we have
	\begin{align}
		\label{e:IQ}
		%
		\norm{Q}{V} = \norm{I - Q}{V}
		.
	\end{align}
	%
	%
	%
	Therefore, for any $v \in V$ and $n \geq 1$,
	\begin{align}
		\norm{v - P_n v}{V}
		\leq
		\norm{v - Q_n v}{V}
		& =
		\norm{(I - Q_n)(v - P_n v)}{V}
		%
		\\
		% 
		& \leq
		\norm{I - Q_n}{V} \norm{v - P_n v}{V}
		=
		\norm{Q_n}{V} \norm{v - P_n v}{V}
		.
	\end{align}
	
	%
	
	
	%
	
	\subsection*{First characterization}
	
	%
	
	Consider the statements
	\begin{align}
		\label{e:1}
		%
		\sup_{v \in V}
		\frac{
			\norm{v - Q_n v}{V}
		}{
			\norm{v - P_n v}{V}
		}
		\to 1
		\quad\text{as}\quad
		n \to \infty
	\end{align}
	and
	\begin{align}
		\label{e:Q1}
		%
		\norm{Q_n}{V} \to 1
		\quad\text{as}\quad
		n \to \infty
		.
	\end{align}
	
	%
	
	Clearly, \eqref{e:Q1} implies \eqref{e:1}.
	%
	Conversely, 
	\eqref{e:1} implies
	$
		\norm{Q_n}{V} = \norm{I - Q_n}{V}
		=
		(1 + o(1))
		\norm{I - P_n}{V}
		=
		(1 + o(1))
		\norm{P_n}{V}
		\to 1
	$
	as $n \to \infty$.
	%
%	Or:
%	if \eqref{e:1} does not hold,
%	e.g.~if
%	\begin{align}
%		\norm{I - Q_n}{V} \norm{v}{V}
%		\geq 
%		\norm{v - Q_n v}{V} 
%		\geq 
%		(1 + \varepsilon) \norm{v - P_n v}{V}
%		\geq 
%		(1 + \varepsilon) \norm{v}{V}
%		,
%	\end{align}
%	then
%	the identity \eqref{e:IQ} implies
%	$\norm{Q_n}{V} \geq 1 + \varepsilon$.
	%
	Therefore, 
	\begin{align}
		\eqref{e:1}
		\Leftrightarrow
		\eqref{e:Q1}
		.
	\end{align}
	
	%
	
	\subsection*{Second characterization}
	
	%
	
	It was shown in [A/2013, Lemma 6.2] that,
	for any $\kappa > 0$,
	the statements
	\begin{align}
		\label{e:A-Qk}
		%
		\norm{Q_n}{V} \leq \kappa^{-1}
	\end{align}
	and
	\begin{align}
		\label{e:A-Vk}
		%
		\forall v' \in V_n:
		\quad
		\sup_{v \in V_n}
		\frac{
			\scalar{v', v}{}
		}{
			\norm{v}{V}
		}
		\geq
		\kappa
		\norm{v'}{V'} 
	\end{align}
	are equivalent.
	%
	%
	The latter statement is a measure of self-duality of $V_n \subset V$.
	%
	%
	In the following, $0 \leq \kappa^*(V_n) \leq 1$ denotes 
	the maximal $\kappa$
	for any closed subspace $V_n \subset V$.
	
	%
	
	\subsection*{Third characterization (special setting)}
	
	%
	
%	Let 
%	$V_J \subset V_{J + 1} \subset H$
%	be another nested sequence of nontrivial finite-dimensional subspaces,
%	whose union is dense in $H$.
%	%
%	Assume
%	the projection $Z_J \from V \to V_J$
%	is orthogonal
%	both
%	in $V$ and $H$.
%	%
%	We abbreviate $v_J := Z_J$ and $v_{-J} := (I - Z_J) v$.
	%
	%
	%
	The setting of [KB/2020]
	is atypical for Gelfand triples
	in that 
	\begin{align}
		\norm{\cdot}{V} \sim \norm{\cdot}{H}
		\ 
		\text{are equivalent norms}
		.
	\end{align}
	%
	We assume this from now on.
	%
	%
	In particular,
	\eqref{e:A-Qk}
	holds uniformly in $n$ with \emph{some} $\kappa$,
	because
	$c \norm{Q_n v}{V} \leq \norm{Q_n v}{H} \leq \norm{v}{H} \leq C \norm{v}{V}$,
	and therefore the arguments in the introduction show that
	\eqref{e:P} also holds for $Q_n$.
	%
%	Consequently, 
%	for each $J$,
%	we have 
%	\begin{align}
%		\label{e:QJ}
%		%
%		\norm{(P_n - Q_n) v_J}{V}
%		\leq
%		o_n^J
%		\norm{v_J}{V}
%		\quad
%		\text{uniformly in $v$}
%	\end{align}
%	where $o_n^J \searrow 0$ as $n \to \infty$.
	%
	%
	%
	In this context, [KB/2020]
	propose a condition,
	which translates to the following:
	%
	Suppose $\{ \varphi_j \}_{j \geq 1}$
	is an orthonormal basis for $H$
	that is also orthogonal in $V$.
	%
	Then:
	\begin{align}
		\label{e:III}
		%
		\text{
			$w_j := \norm{\varphi_j}{V} \to a$ as $j \to \infty$,
			for some $a > 0$.
		}
	\end{align}

	%
	
	For each $J \geq 1$,
	define the subspace $W_J \subset V$ as the span of 
	$\varphi_j$ with $j \leq J$,
	and
	$W_{-J} \subset V$ as the span of those with $j > J$.
	%
	%
	Those subspace are self-dual, 
	because
	$\kappa = 1$ in \eqref{e:A-Vk}
	for 
	any $V_n$ that is a linear span 
	of finitely many $\varphi_j$,
	and $P_n = Q_n$ on such subspaces.
	%
	For this reason alone, 
	\begin{align}
		\text{
			\eqref{e:Q1}
			does not imply
			\eqref{e:III}
			.
		}
	\end{align}
	%
	%
	What is more, the subspace $V_n$
	could be at some angle to all such subspaces.
	%
	%
	For example, take an arbitrary nonzero sequence $c \in \ell_2(\IN)$.
	%
	%
	Define $V_1$ as the span of $\varphi := \sum_j c_j \varphi_j$.
	%
	%
	Then
	\begin{align}
		\textstyle
		\norm{\varphi}{V'}^2
		=
		\sum_j c_j^2 w_j^{-2}
		,
		\quad
		\norm{\varphi}{H}^2
		=
		\sum_j c_j^2
		,
		\quad
		\norm{\varphi}{V}^2
		=
		\sum_j c_j^2 w_j^2
		.
	\end{align}
	%
	%
	Therefore, the $\kappa$ in \eqref{e:A-Vk}
	is at best
	\begin{align}
		\kappa^\star
		(V_1)
		:=
		\frac{
			\norm{\varphi}{H}^2
		}{
			\norm{\varphi}{V'}
			\norm{\varphi}{V}
		}
		\geq \kappa
		.
	\end{align}
	%
	%
	Suppose only $c_1 \neq 0$ and $c_2 \neq 0$
	and
	$w_2 = w_3 = \ldots$.
	%
	Then, for large $w_1$ we have 
	$
		\kappa^\star(V_1) \approx \frac{1}{w_1}(\frac{c_1}{c_2} + \frac{c_2}{c_1})
	$,
	which can be arbitrarily small.
	%
	Condition \eqref{e:III}
	is meant to moderate this behavior.
	%
	Indeed, 
	condition \eqref{e:III} 
	constrains the spread of $w_j$,
	i.e.~it is equivalent to
	\begin{align}
		\omega_{-J}
		:=
		\frac{
			\sup_{j > J} w_j
		}{
			\inf_{j > J} w_j
		}
		\to 1
		\quad\text{as}\quad 
		J \to \infty
		.
	\end{align}
	%
	It follows that
	$
		\kappa^\star(W_{-J}) \geq 1 / \omega_{-J}
		\to 1
	$
	as
	$J \to \infty$.
	%
	%
	
	
	
	
	
	
%	Consider now the following quasi-orthogonality condition
%	\begin{align}
%		\label{e:Q-J}
%		%
%		\norm{(P_n - Q_n) v_{-J}}{V}
%		\leq
%		o_J
%		\norm{v_{-J}}{V}
%		\quad
%		\text{uniformly in $v$ and $n$}
%		,
%	\end{align}
%	where $o_J \searrow 0$ as $J \to \infty$.
%	%
%	%
%	In other words
%	(but without proof), 
%	the equivalence $\norm{\cdot}{V} \sim \norm{\cdot}{H}$ tends to equality
%	outside of $V_J$ as $J \to \infty$.
%	
%	%
%	
%	\textbf{Claim 1}:
%	$\eqref{e:Q-J} \Rightarrow \eqref{e:Q1}$.
%	%
%	%
%	Indeed, for any given $\varepsilon > 0$,
%	we can choose $J$
%	and then $n$ large enough 
%	such that $o_n^J \leq o_J \leq \varepsilon$.
%	%
%	Take any $v \in V$.
%	%
%	Then, using \eqref{e:QJ} and \eqref{e:Q-J},
%	\begin{align}
%		\norm{Q_n v}{V}
%		\leq
%		\norm{(P_n - Q_n) v_J}{V}
%		+
%		\norm{(P_n - Q_n) v_{-J}}{V}
%		+
%		\norm{P_n v}{V}
%		\leq
%		(1 + 2 \varepsilon) \norm{v}{V}
%		.
%	\end{align}
%	%
%	Whence \eqref{e:Q1}.
%	
%	%
%	
%	\textbf{Claim 2}:
%	$\eqref{e:Q-J} \Leftarrow \eqref{e:Q1}$.
%	%
%	%
%	
%	If \eqref{e:J} fails, then
%	for arbitrary large $J$
%	for every $\varepsilon > 0$
%	there exist $v$ and $n$ 
%	such that
%	$\norm{Q_n v_{-J}}{V} > (1 + \varepsilon) \norm{v_{-J}}{V}$.
	 
	

	\section*{References}

	[KB/2020]
	K.~Kirchner, D.~Bolin.
	Necessary and sufficient conditions for asymptotically optimal linear prediction of random fields on compact metric spaces.
	arXiv:2005.08904.
	
	[S/2006]
	D.~B.~Szyld.
	The many proofs of an identity on the norm of oblique projections.
	Numer.~Algorithms, 42(2006), 309--323.
	
	[A/2013]
	R.~Andreev.
	Stability of sparse space--time finite element discretizations of linear parabolic
	evolution equations.
	IMAJNA, 33(2013), 242--260.
	
	\vfill
	\hfill
	RA, \today
\end{document}
