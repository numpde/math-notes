\documentclass[12pt]{article}

\usepackage{amsmath, amsfonts}

%
\usepackage{fullpage}
\setlength{\parindent}{0cm}

% For editing purposes:
%\usepackage[margin=10pt]{geometry}

% Macros
\newcommand{\IP}{\mathbb{P}}
\newcommand{\IE}{\mathbb{E}}


\begin{document}
	
	\section*{Black--Scholes: derivation}
	
	Let $S_t$ be the price of the underlying at time $t$.
	Suppose that over a small time period $\Delta t$,
	the price moves 
	with the following probabilities:
	\begin{align}
		p_\pm
		:=
		\IP[
			S_{t + \Delta t} = \gamma_\pm S_t
		]
		\quad\text{where}\quad
		\gamma_\pm := 1 + \mu \Delta t \pm \sigma_\pm \sqrt{\Delta t}
		.
	\end{align}
	
	Then, over $n$ such periods
	with independent price shifts,
	we have
	\begin{align}
		\IP[
			S_{t + n \Delta t}
			=
			\gamma_+^{k}
			\gamma_-^{n - k}
			S_t
		]
		=
		\binom{n}{k}
		p_+^{k}
		p_-^{n - k}
		.
	\end{align}
	
	The expected price is thus
	\begin{align}
		\IE[S_{t + n \Delta t}]
		& =
		\sum_{k = 0}^n
		\binom{n}{k}
		(\gamma_+ p_+)^{k}
		(\gamma_- p_-)^{n - k}
		S_t
		\\ & =
		(
			p_+ \gamma_+
			+
			p_- \gamma_-
		)^n
		S_t
		\\ & =
		(
			1 + \mu \Delta t
			+
			(p_+ \sigma_+  -  p_- \sigma_-) \sqrt{\Delta t}
		)^n
		S_t
		.
	\end{align}
	
	Now we couple $\Delta t = 1 / n$.
	%
	Define $\sigma^2 := \sigma_+ \sigma_-$.
	%
	We assume that
	\begin{align}
		p_+ \sigma_+
		=
		p_- \sigma_-
	\end{align}
	so that
	the following are finite, as $n \to \infty$,
	\begin{align}
		\IE[S_{t + 1} / S_t] 
		\to 
		e^\mu
		%
		\quad\text{and}\quad
		%
		\IE[S_{t + 1}^2 / S_t^2] 
		\to
		e^{2 \mu + \sigma^2}
		.
	\end{align}
	
	This suggests 
	the following expressions that can be used to estimate $\mu$ and $\sigma^2$:
	\begin{align}
		\mu =
		\log \IE[S_{t + 1} / S_t]
		%
		\quad\text{and}\quad
		%
		\sigma^2
		=
		\log \frac{\IE[S_{t + 1}^2 / S_t^2]}{\IE[S_{t + 1} / S_t]^2}
		.
	\end{align}
	
	Let $V(s, t)$ denote the price of 
	the option at time $t$ when the price
	of the underlying is $s$.
	%
	Consider the portfolio
	$\Pi = \alpha S + B$,
	consisting of the underlying $S$ and riskless $B$.
	%
	We require
	$\Pi_{t + \Delta t} = \alpha S_{t + \Delta t} + (1 + r \Delta t) B$,
	where $r$ is the risk-free rate of return.
	%
	The two possibilities $S_{t + \Delta t} = \gamma_\pm S_t$
	determine $\alpha$ and $B$,
	such that
	\begin{align}
		\Pi_t = 
		\frac{1}{1 + r \Delta t}
		\left(
			q_+
			V(\gamma_+ S_t, t + \Delta t)
			+
			q_-
			V(\gamma_- S_t, t + \Delta t)
		\right)
		\quad\text{with}\quad
		q_\pm
		=
		\frac{\sigma_\mp \pm (1 + r \Delta t)}{\sigma_+ + \sigma_-}
		.
	\end{align}
	
	The no-arbitrage argument gives $V(S_t, t) = \Pi_t$.
	%
	Thus, we have the relation
	\begin{align}
		V(s, t)
		=
		\frac{1}{1 + r \Delta t}
		\sum_\pm
		q_\pm V(\gamma_\pm s, t + \Delta t)
		.
	\end{align}
	
	Note that
	\begin{align}
		(\gamma_+ - 1) q_+ + (\gamma_- - 1) q_-
		=
		r \Delta t
	\end{align}
	and
	\begin{align}
		(\gamma_+ - 1)^2 q_+ + (\gamma_- - 1)^2 q_-
		=
		(\sigma^2 + \mathcal{O}(\sqrt{\Delta t})) \Delta t
		.
	\end{align}
	
	Therefore, Taylor expansion of $V$ in the first argument gives
	\begin{align}
		\frac{V(s, t) - V(s, t + \Delta t)}{\Delta t}
		\approx
		-r V(s, t)
		+
		r 
		s
		\frac{\partial V}{\partial s}(s, t + \Delta t)
		+
		\frac12
		\sigma^2
		s^2
		\frac{\partial^2 V}{\partial s^2}(s, t + \Delta t)
		,
	\end{align}
	where the error is $\mathcal{O}(\sqrt{\Delta t})$.
	%
	Whence
	ensues the Black--Scholes equation:
	\begin{align}
		\frac{\partial V}{\partial t}
		+
		r
		s
		\frac{\partial V}{\partial s}
		+
		\frac12
		\sigma^2 s^2
		\frac{\partial^2 V}{\partial s^2}	
		-
		r V
		= 0
		.	
	\end{align}
	
	\vfill
	\hfill
	RA, \today
\end{document}
